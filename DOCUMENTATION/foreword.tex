\foreword

\SES is a programme package for the simulation of elastic wave
propagation and waveform inversion in a spherical section. The package is based on a spectral-element discretisation of
the seismic wave equation combined with adjoint techniques.\\[5pt]
\SES supports 3D heterogeneous visco-elastic rheologies with radial anisotropy. Anisotropic perfectly matched layers
are implemented to avoid reflections from the unphysical boundaries of the spherical section.\\[5pt]
\SES operates in the natural spherical coordinates, which is untypical for spectral-element approaches. The advantages are a
compact programme code, fast computations for spherical sections that are sufficiently far
from the poles and the core, and the easy implementation of 3D models.\\[5pt]
\SES is fully parallelised, meaning that the computational domain is partitioned into subdomains, each of which is assigned to one compute core. Communication between subdomains is based on MPI.\\[5pt]
\SES has been developed for continental-scale full seismic waveform
inversion. It is, however, applicable to a wide range of local- to continental-scale wave propagation problems.\\[5pt]
\SES is deliberately puristic. This is intended to (1) make the code
easily adaptable to particular problems, (2) facilitate the
implementation of 3D models, (3) reduce the likelihood of
programming errors, and (4) allow for an easy adaptation to new hardware architectures. The last point becomes particularly relevant in times when hardware architecture changes rapidly.\\[5pt]
This tutorial is split into a description of the programme code and
an introduction to the mathematical background of \SES. Reading the
mathematical part is not required to successfully run \SES. It is,
however, strongly recommended. The chance of using \SES incorrectly
or inefficiently is high when its mathematical background is not
known. This is true for any numerical method.\\[5pt]
The description of the programme package is centred around
realistic examples that a new user may want to reproduce in order to
become familiar with \SES.\\[5pt]
\textbf{\SES is hosted on} \href{https://github.com}{Github} \textbf{and can be obtained from} \href{https://github.com/echolite/ses3d}{https://github.com/echolite/ses3d}.


\vspace{\baselineskip}
\begin{flushright}\noindent
Zurich, February 2014\hfill {\it Andreas Fichtner}\\
\end{flushright}
