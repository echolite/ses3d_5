\chapter{The seismic wave equation}\label{C:AnalyticalSetup}

This chapter is concerned with the mathematical description of
seismic wave propagation in the Earth and the derivation of
equations that can be solved numerically in an efficient way.
Following a brief review of the elastic wave equation and its
subsidiary conditions in section \ref{S:wave_equation}, we elaborate
on the simulation of anisotropy (section \ref{S:anisotropy}) and
attenuation (section \ref{S:attenuation}). We also cover the
implementation and analysis of absorbing boundaries within this
chapter on the analytical setup, because their properties are
relatively independent from a particular numerical scheme.

\section{The elastic wave equation}\label{S:wave_equation}

The propagation of seismic waves in the Earth can be modelled with
the elastic wave equation
\begin{equation}\label{E:into_waveeq001}
\rho(\w{x})\ddot{\w{u}}(\w{x}, t) - \nabla\cdot\s{\sigma}(\w{x}, t)
= \w{f}(\w{x}, t)\,,
\end{equation}
that relates the displacement field $\w{u}$ to the mass density
$\rho$, the stress tensor $\s{\sigma}$ and an external force density
$\w{f}$ (e.g. Dahlen \& Tromp, 1998; Kennett, 2001; Aki \& Richards, 2002). The elastic wave
equation as presented above is a linearised version of the momentum
balance equation, i.e., of Newton's second law. It is valid under
the assumption that deviations from the reference configuration of
the Earth are small. Furthermore, the rotation of the Earth and its
self-gravitation are omitted. These effects are negligible when
oscillation periods are shorter than $\sim 200$ s. At the surface
$\delta\oplus$ of the Earth $\oplus$ the normal components of the
stress tensor $\s{\sigma}$ vanish:
\begin{equation}\label{E:into_waveeq002}
\s{\sigma}\cdot\w{e}_r |_{\w{x}\in\partial\oplus} = \w{0}\,.
\end{equation}
Equation (\ref{E:into_waveeq002}) is the free surface boundary
condition. Both the displacement field $\w{u}$ and the velocity
field $\w{v}=\dot{\w{u}}$ are required to vanish prior to $t=t_0$
when the external force $\w{f}$ starts to act, i.e.,
\begin{equation}\label{E:into_waveeq003}
\w{u}|_{t<t_0}=\w{v}|_{t<t_0}=\w{0}\,.
\end{equation}
For convenience we will mostly choose $t_0=0$. To obtain a complete
set of equations, the stress tensor $\s{\sigma}$ has to be related
to the displacement field $\w{u}$. It is usually assumed that the
rheology is visco-elastic, meaning that the current stress tensor
$\s{\sigma}$ depends linearly on the history of the strain tensor
$\s{\epsilon}=\frac{1}{2}(\nabla \w{u} + \nabla \w{u}^T)$:
\begin{equation}\label{E:intro_waveeq004}
\s{\sigma}(\w{x}, t) = \int\limits_{-\infty}^{\infty}
\w{\dot{C}}(\w{x}, t-t'):\s{\epsilon}(\w{x}, t')\,dt'\,.
\end{equation}
The fourth order tensor $\w{C}$ is the elastic tensor. In the case
of a perfectly elastic medium the elastic tensor is of the form
$\w{C}(\w{x},t)=\w{C}(\w{x})\,H(t)$, where $H$ is the Heaviside
function. We then have $\s{\sigma}=\w{C}(\w{x}):\s{\epsilon}(\w{x},
t)$. The symmetry of $\s{\epsilon}$, the conservation of angular
momentum and the relation of $\w{C}$ to the internal energy (e.g.
Dahlen \& Tromp, 1998) require that the components of $\w{C}$
satisfy the following symmetry relations:
\begin{equation}\label{E:intro_waveeq005}
C_{ijkl}=C_{klij}=C_{jikl}\,.
\end{equation}
Moreover, the elastic tensor is causal:
\begin{equation}\label{E:into_waveeq006}
\w{C}(t)|_{t<t_0}=\w{0}\,.
\end{equation}
Equation (\ref{E:intro_waveeq004}) is $-$ unlike the wave equation
within its limits of validity $-$ not a fundamental law of physics
but an empirical relation that has been found to describe a wide
range of phenomena very well. It can be regarded as a linearisation
of a more general non-linear constitutive relation. Its validity is,
in this sense, restricted to scenarios where the strain tensor
$\s{\epsilon}$ is a small quantity. The particular choices of
$\w{C}$ and its time dependence will be the subjects of the
following sections on attenuation and anisotropy.

\section{Anisotropy}\label{S:anisotropy} 
Anisotropy is the dependence of the elastic tensor on the orientation of the
coordinate system. Its most direct seismological expression is the
dependence of seismic velocities on the propagation and polarisation
directions of elastic waves. It is thought to play a major role in the Earth's crust and upper mantle.\\
Besides the splitting of shear waves, the Love wave-Rayleigh wave
discrepancy is one of the principal seismic observations that is
directly related to anisotropy: A Love wave and a Rayleigh wave
travelling in the same direction usually exhibit different wave
speeds as a consequence of their different polarisations. This led
to the inclusion of anisotropy with radial symmetry axis in the
global reference model PREM (Dziewonski \& Anderson, 1981). In this
model the anisotropy
is limited to the upper 220 km.\\
Guided by these observations, we decided to implement anisotropy
with radial symmetry axis. For such a medium, there are only 5
independent elastic tensor components that are different from zero.
They can be summarised in a $6\times 6$ matrix (e.g. Babuska \& Cara, 1991):
\begin{align}\label{E:setup15}
&\begin{pmatrix}
C_{r r r r} & C_{r r \phi \phi} & C_{r r \theta \theta} & C_{r r \phi \theta} & C_{r r r \theta} & C_{r r r \phi} \\
C_{\phi \phi r r} & C_{\phi \phi \phi \phi} & C_{\phi \phi \theta \theta} & C_{\phi \phi \phi \theta} & C_{\phi \phi r \theta} & C_{\phi \phi r \phi} \\
C_{\theta \theta r r} & C_{\theta \theta \phi \phi} & C_{\theta \theta \theta \theta} & C_{\theta \theta \phi \theta} & C_{\theta \theta r \theta} & C_{\theta \theta r \phi} \\
C_{\phi \theta r r} & C_{\phi \theta \phi \phi} & C_{\phi \theta \theta \theta} & C_{\phi \theta \phi \theta} & C_{\phi \theta r \theta} & C_{\phi \theta r \phi} \\
C_{r \theta r r} & C_{r \theta \phi \phi} & C_{r \theta \theta \theta} & C_{r \theta \phi \theta} & C_{r \theta r \theta} & C_{r \theta r \phi} \\
C_{r \phi r r} & C_{r \phi \phi \phi} & C_{r \phi \theta \theta} &
C_{r \phi \phi \theta} & C_{r \phi r \theta} & C_{r \phi r \phi}
\end{pmatrix}\notag\\
=&
\begin{pmatrix}
\lambda+2\mu & \lambda+c & \lambda+c & 0 & 0 & 0 \\
\lambda+c & \lambda+2\mu+a & \lambda+a & 0 & 0 & 0 \\
\lambda+c & \lambda+a & \lambda+2\mu+a & 0 & 0 & 0 \\
0 & 0 & 0 & \mu & 0 & 0 \\
0 & 0 & 0 & 0 & \mu+b & 0 \\
0 & 0 & 0 & 0 & 0 & \mu+b
\end{pmatrix}
\end{align}
Love (1892) proposed an alternative parametrisation
\begin{align}\label{E:setup15b}
&\begin{pmatrix}
C_{r r r r} & C_{r r \phi \phi} & C_{r r \theta \theta} & C_{r r \phi \theta} & C_{r r r \theta} & C_{r r r \phi} \\
C_{\phi \phi r r} & C_{\phi \phi \phi \phi} & C_{\phi \phi \theta \theta} & C_{\phi \phi \phi \theta} & C_{\phi \phi r \theta} & C_{\phi \phi r \phi} \\
C_{\theta \theta r r} & C_{\theta \theta \phi \phi} & C_{\theta \theta \theta \theta} & C_{\theta \theta \phi \theta} & C_{\theta \theta r \theta} & C_{\theta \theta r \phi} \\
C_{\phi \theta r r} & C_{\phi \theta \phi \phi} & C_{\phi \theta \theta \theta} & C_{\phi \theta \phi \theta} & C_{\phi \theta r \theta} & C_{\phi \theta r \phi} \\
C_{r \theta r r} & C_{r \theta \phi \phi} & C_{r \theta \theta \theta} & C_{r \theta \phi \theta} & C_{r \theta r \theta} & C_{r \theta r \phi} \\
C_{r \phi r r} & C_{r \phi \phi \phi} & C_{r \phi \theta \theta} &
C_{r \phi \phi \theta} & C_{r \phi r \theta} & C_{r \phi r \phi}
\end{pmatrix}\notag\\
=&
\begin{pmatrix}
C_L & F_L & F_L & 0 & 0 & 0 \\
F_L & A_L & A_L-2N_L & 0 & 0 & 0 \\
F_L & A_L-2N_L & A_L & 0 & 0 & 0 \\
0 & 0 & 0 & N_L & 0 & 0 \\
0 & 0 & 0 & 0 & L_L & 0 \\
0 & 0 & 0 & 0 & 0 & L_L
\end{pmatrix}
\end{align}
We chose to use the parametrisation from equation (\ref{E:setup15})
because it easily allows us to model isotropy by simply setting
$a=b=c=0$. Remember, that all components of the elastic tensor are
functions of time, $t$, and space, $\w{x}$. Density, $\rho$ and the
elastic parameters $\lambda, \mu, a, b$ and $c$ can be related to
the wave speeds of SH, SV, PH and PV waves:
\begin{equation}\label{E:svsh17}
\vsv=\sqrt{\frac{\mu+b}{\rho}}\,,\quad
\vsh=\sqrt{\frac{\mu}{\rho}}\,,\quad
\vpv=\sqrt{\frac{\lambda+2\mu}{\rho}}\,,\quad
\vph=\sqrt{\frac{\lambda+2\mu+a}{\rho}}\,.
\end{equation}
The velocities of SH, SV, PH and PV waves specify only two of the
three additional elastic parameters necessary for anisotropy with
radial symmetry axis, namely $a$ and $b$. Also radially propagating
S waves do not allow us to find $c$, because they propagate with the
velocity $\sqrt{(\mu+b)/\rho}$, just as SV waves. In fact, using
plane waves, $c$ can be determined only from P waves that do not
travel in exactly radial or horizontal directions. Following
Takeuchi \& Saito (1972) and Dziewonski \& Anderson (1981) we
incorporate $c$ into a dimensionless parameter
\begin{equation}\label{E:pvph007}
\eta:=(\lambda+c)/(\lambda+a)\,.
\end{equation}
