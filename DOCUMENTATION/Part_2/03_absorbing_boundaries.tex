
\chapter{Absorbing boundaries}\label{S:absbound_theo}

Restricting the considered spatial domain to only a part of the true physical
domain in the interest of computational efficiency, introduces
unrealistic boundaries. If not treated adequately, the reflections
from the artificial boundaries dominate the numerical error. The
most widely used solutions for this problem fall into two
categories: \emph{absorbing boundary conditions} and \emph{absorbing
boundary layers}.\\
Absorbing boundary conditions are usually based on paraxial
approximations of the wave equation. Early applications of this
technique to finite-difference modelling can be found in Clayton \&
Engquist (1977) or in Stacey (1988). Along the artificial boundary the wave
equation is replaced by one of its paraxial approximations of order
$n$, typically not higher than one or two. The reflection
coefficient then behaves as $(\sin \phi)^n$, where $\phi$ is the
angle of incidence. Absorbing boundary conditions therefore become
inefficient for large angles of incidence in general and for surface
waves in particular. Moreover, they suffer from numerical stability
problems.\\
Absorbing boundary layers are regions near the unphysical boundaries
where the wave field is artificially attenuated in order to prevent
reflections. Cerjan et al. (1985) proposed to multiply the wave
field in each time step with a Gaussian damping function. While this
technique proves to be efficient for finite-difference methods, it
leads to unacceptably large boundary layers when high-order methods
such as the spectral-element method are used. Robertsson et al.
(1994) suggested that the absorption could be improved through
physical dissipation, i.e., a very low $Q$ inside the boundary
layers. This, however, leads to reflections from the boundary layer
itself because low $Q$ values effectively change the elastic
properties of the medium.\\
A comparatively efficient boundary layer technique was introduced by
B\'{e}renger (1994), who proposed to modify the electrodynamic wave
equation inside a \emph{perfectly matched layer} (PML) such that the
solutions decay exponentially with distance, without producing
reflections from the boundary between the regular medium and the
PML. Though originally designed for first-order systems of
differential equations - such as the elastodynamic equations in
stress-velocity formulation (see e.g. Collino \& Tsogka, 2001 or
Festa \& Vilotte, 2005) - it has been shown that the method can be
extended to second-order systems (Komatitsch \& Tromp, 2003). The
classical PML approach leads to a new and generally larger system of
differential equations. Therefore, existing codes have to be
substantially modified. This complication can be avoided by using
\emph{anisotropic perfectly matched layers} (APML), studied for
example by Teixeira \& Chew (1997) and Zheng \& Huang (1997). \SES implements the APML, which are described in the following paragraphs.\\[5pt]
The elastic wave equation in the frequency domain is given by
\begin{subequations}\label{E:apml001}
\begin{align}
\w{i}\omega \rho \w{v}(\w{x},\omega) &= \nabla\cdot\s{\sigma}(\w{x},
\omega) + \w{f}(\w{x},\omega)\,,\label{E:apml001a}\\
\w{i}\omega \s{\sigma}(\w{x},\omega) &=
\s{\Xi}(\w{x},\omega):\nabla \w{v}(\w{x},\omega)\,,\label{E:apml001b}\\
\w{i}\omega \w{u}(\w{x},\omega) &=
\w{v}(\w{x},\omega)\,.\label{E:apml001c}
\end{align}
\end{subequations}
We introduced the symbol $\s{\Xi}$ for the rate of relaxation tensor
$\dot{\w{C}}$ in equations (\ref{E:apml001}). In a homogeneous and
isotropic medium and when external forces are absent, it is known to
have plane wave solutions of the form
\begin{equation}\label{E:apml002}
\w{u}(\w{x},\omega)=\w{A}\,e^{-\w{i}\w{k}(\omega)\cdot \w{x}}\,,
\end{equation}
with either
\begin{subequations}
\begin{align}
&\w{A}\,\, ||\,\, \w{k}(\omega)\quad\text{and}\quad
||\w{k}(\omega)||^2=\frac{\rho\omega^2}{\lambda+2\mu}=\frac{\omega^2}{\vp^2}\,\quad&&\text{(P-wave)}\,\quad\text{or}\,\label{E:apml003}\\
&\w{A}\perp\w{k}(\omega)\quad\text{and}\quad
||\w{k}(\omega)||^2=\frac{\rho\omega^2}{\mu}=\frac{\omega^2}{\vs^2}\,\quad&&\text{(S-wave)}\,.\label{E:apml004}
\end{align}
\end{subequations}
Our goal is to modify (\ref{E:apml001}) such that it allows for
plane wave solutions that decay exponentially with distance, say for
example, in increasing $z$ direction. For this, the cartesian
coordinate $z$ is replaced by a new variable
$\tilde{z}:=\tilde{z}(\w{x},\gamma_z(\w{x}))$, such that the
resulting solutions are
\begin{equation}\label{E:apml005}
\w{u}(\w{x}, \omega)=
\w{A}\,e^{-\w{i}\w{k}(\omega)\cdot\w{x}}\,e^{-f(\gamma_z) z}\,,
\end{equation}
for $z>0$ and some function $f$ that depends on the particular
choice of $\gamma_z$. Choosing $f(\gamma_z)|_{z<0}=0$ will leave the
solution in $z<0$ unchanged. We use the coordinate transformation
\begin{equation}\label{E:apml006}
\tilde{z}:=\int\limits_{0}^{z} \gamma_z(x, y, z') dz'\,.
\end{equation}
Outside the perfectly matched layer, that is in our case for $z<0$,
we require $\tilde{z}=z$ or equivalently $\gamma_z(\w{x})=1$. To
obtain exponentially decaying plane waves also in the remaining
coordinate directions, the transformations $x\to\tilde{x}$ and
$y\to\tilde{y}$ are defined analogously. Replacing the derivatives
$\partial_x$, $\partial_y$ and $\partial_z$ in (\ref{E:apml001a}) by
the derivatives $\partial_{\tilde{x}}$, $\partial_{\tilde{y}}$ and
$\partial_{\tilde{z}}$, gives the following set of equations,
\begin{subequations}\label{E:apml007}
\begin{align}
\w{i}\omega\rho v_x &= \gamma_x^{-1} \partial_x \sigma_{xx} +
\gamma_y^{-1} \partial_y \sigma_{yx} + \gamma_z^{-1} \partial_z
\sigma_{zx}\,,\label{E:apml007a}\\
\w{i}\omega\rho v_y &= \gamma_x^{-1} \partial_x \sigma_{xy} +
\gamma_y^{-1} \partial_y \sigma_{yy} + \gamma_z^{-1} \partial_z
\sigma_{zy}\,,\label{E:apml007b}\\
\w{i}\omega\rho v_z &= \gamma_x^{-1} \partial_x \sigma_{xz} +
\gamma_y^{-1} \partial_y \sigma_{yz} + \gamma_z^{-1} \partial_z
\sigma_{zz}\,.\label{E:apml007c}
\end{align}
\end{subequations}
Note that the divergence in equation (\ref{E:apml001a}) is
interpreted as a left-divergence in equations (\ref{E:apml007}).
Multiplication by $\gamma_x \gamma_y \gamma_z$ yields
\begin{subequations}\label{E:apml008}
\begin{align}
\w{i}\omega \gamma_x \gamma_y \gamma_z v_x &= \gamma_y \gamma_z
\partial_x \sigma_{xx} + \gamma_x \gamma_z \partial_y \sigma_{yx} +
\gamma_x \gamma_y \partial_z \sigma_{zx}\,,\label{E:apml008a}\\
\w{i}\omega \gamma_x \gamma_y \gamma_z v_y &= \gamma_y \gamma_z
\partial_x \sigma_{xy} + \gamma_x \gamma_z \partial_y \sigma_{yy} +
\gamma_x \gamma_y \partial_z \sigma_{zy}\,,\label{E:apml008b}\\
\w{i}\omega \gamma_x \gamma_y \gamma_z v_z &= \gamma_y \gamma_z
\partial_x \sigma_{xz} + \gamma_x \gamma_z \partial_y \sigma_{yz} +
\gamma_x \gamma_y \partial_z \sigma_{zz}\,.\label{E:apml008c}
\end{align}
\end{subequations}
The products $\gamma_i \gamma_j$ on the left-hand sides of
(\ref{E:apml008}) can be placed under the differentiation if
$\gamma_x$ depends only on $x$, $\gamma_y$ only on $y$ and
$\gamma_z$ only on $z$. We may then write the new version of
(\ref{E:apml001a}) as
\begin{equation}\label{E:apml009}
\w{i}\omega \gamma_x \gamma_y \gamma_z \w{v} = \nabla\cdot
\s{\sigma}^{\text{pml}}\,,
\end{equation}
with the definition
\begin{equation}\label{E:apml010}
\s{\sigma}^{\text{pml}} := \gamma_x \gamma_y \gamma_z \left(
  \begin{array}{ccc}
    \gamma_x^{-1} & 0 & 0 \\
    0 & \gamma_y^{-1} & 0 \\
    0 & 0 & \gamma_z^{-1} \\
  \end{array}
\right) \cdot \s{\sigma} = \gamma_x \gamma_y \gamma_z
\s{\Lambda}\cdot\s{\sigma}\,.
\end{equation}
Since (\ref{E:apml001a}) and (\ref{E:apml009}) are very similar, the
anisotropic perfectly matched layers can be implemented with
relative ease by slightly modifying pre-existing codes. It is
noteworthy at this point that $\s{\sigma}^{\text{pml}}$ is in
general not symmetric.\\
In the next step we consider the constitutive relation. Based on
(\ref{E:apml001b}) and (\ref{E:apml001c}) we find
\begin{equation}\label{E:apml011}
\sigma_{ij}^{\text{pml}} = \gamma_x \gamma_y \gamma_z \gamma_i^{-1}
\sigma_{ij} = \gamma_x \gamma_y \gamma_z \gamma_i^{-1}
\sum_{k,l=1}^{3} \Xi_{ijkl} \partial_{x_k} u_l\,.
\end{equation}
Again replacing $\partial_x$, $\partial_y$ and $\partial_z$ by
$\partial_{\tilde{x}}$, $\partial_{\tilde{y}}$ and
$\partial_{\tilde{z}}$, yields
\begin{equation}\label{E:apml012}
\sigma_{ij}^{\text{pml}}=  \gamma_x \gamma_y \gamma_z \gamma_i^{-1}
\sum_{k,l=1}^{3} \Xi_{ijkl} \gamma_k^{-1} \partial_{x_k} u_l\,.
\end{equation}
For convenience we define a new strain tensor
$\s{\epsilon}^{\text{pml}}$ as
\begin{equation}\label{E:apml013}
\epsilon_{kl}^{\text{pml}} :=  \gamma_x \gamma_y \gamma_z
\gamma_k^{-1} \partial_{x_k} u_l\,,
\end{equation}
which finally leads to our modified version of the wave equation:
\begin{subequations}\label{E:apml014}
\begin{align}
\w{i}\omega \rho \gamma_x \gamma_y \gamma_z v_i &= \sum_{j=1}^{3}
\partial_{x_j} \sigma_{ji}^{\text{pml}}\,,\label{E:apml014a}\\
\gamma_i \sigma_{ij}^{\text{pml}} &= \sum_{k,l=1}^{3} \Xi_{ijkl}
\epsilon_{kl}^{\text{pml}}\,,\label{E:apml014b}\\
\epsilon_{kl}^{\text{pml}} &=  \gamma_x \gamma_y \gamma_z
\gamma_k^{-1} \partial_{x_k} u_l\,.\label{E:apml014c}
\end{align}
\end{subequations}
The source term in (\ref{E:apml014}) is omitted because the PML
region will usually be chosen such that the sources are inside the
regular medium and not inside the PML.\\[10pt]
To demonstrate that exponentially decaying plane waves are indeed
solutions of (\ref{E:apml014}) we consider a homogeneous and
isotropic medium. For simplicity, we restrict the analysis to the
case $\gamma_x=\gamma_y=1$, $\gamma_z = const \neq 1$. Assuming a
plane wave solution of the form
\begin{equation}\label{E:apml015}
\w{u}(\w{x},\omega)= \w{A}\,e^{-\w{i}(k_x x + k_z z ) }\,,
\end{equation}
the problem reduces to finding the dispersion relation
$\w{k}=\w{k}(\omega)$. The Christoffel equation resulting from the
combination of (\ref{E:apml014}) with (\ref{E:apml015}) is
\begin{equation}\label{E:apml016}
\left(
  \begin{array}{ccc}
    \vp^2 k_x^2 +\vs^2 k_z^2 \gamma_z^{-2} & 0 & (\vp^2-\vs^2)(k_x k_z \gamma_z^{-1}) \\
    0 & \vs^2 (k_x^2 + k_z^2 \gamma_z^{-2}) & 0 \\
    (\vp^2-\vs^2)(k_x k_z \gamma_z^{-1}) & 0 & \vp^2 k_z^2 \gamma_z^{-2} + \vs^2 k_x^2 \\
  \end{array}
\right) \left(
  \begin{array}{c}
    A_x \\
    A_y \\
    A_z \\
  \end{array}
\right)
 = \omega^2 \left(
                     \begin{array}{c}
                       A_x \\
                       A_y \\
                       A_z \\
                     \end{array}
                   \right)\,.
\end{equation}
Non-trivial solutions exist only if either
\begin{subequations}\label{E:apml017}
\begin{equation}\label{E:apml018}
k_x^2+k_z^2 \gamma_z^{-2} = \omega^2 \vp^{-2}\,
\end{equation}
or
\begin{equation}\label{E:apml019}
k_x^2+k_z^2 \gamma_z^{-2} = \omega^2 \vs^{-2}\,
\end{equation}
\end{subequations}
are satisfied. Equations (\ref{E:apml018}) and (\ref{E:apml019}) are
the dispersion relations inside the PML. The resulting plane wave
solutions are
\begin{subequations}\label{E:apml020}
\begin{equation}
\w{u}_s(\w{x},\omega)=\w{A}_s\,e^{-\w{i} \omega
(x\,\sin\phi+\gamma_z z\,\cos\phi)/\vs}\,,\quad \w{A}_s\perp (k_x,
0, k_z\gamma_z^{-1})^T\,,
\end{equation}
for the S wave and
\begin{equation}
\w{u}_p(\w{x},\omega)=\w{A}_p\,e^{-\w{i} \omega
(x\,\sin\phi+\gamma_z z\,\cos\phi)/\vp}\,,\quad \w{A}_p\,||\, (k_x,
0, k_z\gamma_z^{-1})^T\,,
\end{equation}
\end{subequations}
for the P wave. The variable $\phi$ denotes the angle of incidence.
One may now define the coordinate stretching variable $\gamma_z$.
There are in principle no restrictions other than $\gamma_z=1$
outside the PML. We use
\begin{equation}\label{E:apml021}
\gamma_z(\w{x}):=1+\frac{a_z}{\w{i}\omega}\,.
\end{equation}
Inserting (\ref{E:apml021}) into (\ref{E:apml020}) yields
\begin{equation}\label{E:apml022}
\w{u}_{s/p}(\w{x},\omega)=\w{A}_{s/p}\,e^{-\w{i} \omega (x\,\sin\phi
+
z\,\cos\phi)/v_\text{\tiny{S/P}}}\,e^{-(a_z\,z\cos\phi)/v_\text{\tiny{S/P}}}\,.
\end{equation}
The exponential decay is therefore frequency-independent in the case
of incident body waves. A frequency-dependent decay can be expected
for dispersive waves such as surface waves. However, waves with an
angle of incidence, $\phi$, close to $\pm \pi/2$ will decay slower
than waves with an angle of incidence close to $0$. It should be
noted that the example of a homogeneous and isotropic medium with
constant $\gamma_z$ is oversimplified. Since analytical solutions
for more complex models are usually unavailable, the efficiency of
the APML
technique must be tested numerically.\\[5pt]
With the exception of the corners of the
model\footnote{\textsf{There is, to the best of my knowledge, no PML
variant where the corners are treated adequately. In the corners two
or more PMLs overlap, and it is not immediately clear what the
consequences are. In fact, numerical experiments show that the
instability tends to start in the corners. This suggests that PML
methods might be improved substantially through the construction of
absorbing corners where elastic waves do indeed decay.}}, the
damping regions at the $x$, $y$ and $z$ boundaries do not overlap.
We therefore find
\begin{equation}\label{E:apml023}
a_i a_j = a_i^2 \delta_{ij}
\end{equation}
and
\begin{equation}\label{E:apml024}
\gamma_x \gamma_y \gamma_z = \frac{\w{i}\omega + a_x + a_y +
a_z}{\w{i}\omega}\,.
\end{equation}
The resulting equations of motion in the frequency domain are now
\begin{subequations}\label{E:apml025}
\begin{align}
\w{i}\omega\rho\,(\w{i}\omega + a_x + a_y + a_z)\, u_i &=
\sum_{j=1}^{3} \partial_{x_j}
\sigma_{ji}^{\text{pml}}\,,\label{E:apml025a}\\
(\w{i}\omega + a_i)\,\sigma_{ij}^{\text{pml}} &= \sum_{k,l=1}^{3}
\w{i}\omega\,\Xi_{ijkl} \epsilon_{kl}^{\text{pml}}\,,\label{E:apml025b}\\
\w{i}\omega\,\epsilon_{kl}^{\text{pml}} &=(\w{i}\omega + a_x + a_y +
a_z - a_k)\,\partial_{x_k} u_l\,.\label{E:apml025c}
\end{align}
\end{subequations}
Transforming into the time domain, finally yields
\begin{subequations}\label{E:apml026}
\begin{align}
\rho\,\partial_t^2 u_i + \rho\,(a_x+a_y+a_z) \partial_t u_i &=
\sum_{j=1}^{3} \partial_{x_j}
\sigma_{ji}^{\text{pml}}\,,\label{E:apml026a}\\
\partial_t \sigma_{ij}^{\text{pml}} + a_i \sigma_{ij}^{\text{pml}}
&= \sum_{k,l=1}^{3} \Xi_{ijkl}
*\partial_t{\epsilon}_{kl}^{\text{pml}}\,,\label{E:apml026b}\\
\partial_t{\epsilon}_{kl}^{\text{pml}} &= \partial_t \partial_{x_k} u_l +
(a_x+a_y+a_z-a_k)\,\partial_{x_k} u_l\,,\label{E:apml026c}
\end{align}
\end{subequations}
or in tensor notation:
\begin{subequations}\label{E:apml027}
\begin{align}
\rho\,\partial_t^2\w{u}+\rho\,\text{tr}(\w{a})\,\partial_t \w{u} &=
\nabla\cdot\s{\sigma}^{\text{pml}}\,,\label{E:apml027a}\\
\partial_t\s{\sigma}^{\text{pml}} +
\w{a}\cdot\s{\sigma}^{\text{pml}} &=\int\limits_{-\infty}^{t}
\s{\Xi}(t-\tau):\partial_t\s{\epsilon}^{\text{pml}}(\tau)\,d\tau\,,\label{E:apml027b}\\
\partial_t\s{\epsilon}^{\text{pml}} &= \partial_t \nabla\w{u} +
(\w{I}\,\text{tr}(\w{a})-\w{a})\cdot\nabla
\w{u}\,,\label{E:apml027c}
\end{align}
\end{subequations}
where $\w{a}$ is defined as
\begin{equation}\label{E:apml028}
\w{a}:=\left(
         \begin{array}{ccc}
           a_x & 0 & 0 \\
           0 & a_y & 0 \\
           0 & 0 & a_z \\
         \end{array}
       \right)\,.
\end{equation}
The gradients in equation (\ref{E:apml027c}) are left-gradients. The
choice of $\gamma_i$ as inversely proportional to $\omega$ leads to
relatively simple differential equations that can be marched forward
in time explicitly. Different definitions of $\gamma_i$ generally
result in temporal convolutions and therefore lead to
integro-differential equations that are more difficult to solve, at
least in the time domain.
