\chapter{Analytical setup}\label{C:AnalyticalSetup}

This chapter is concerned with the mathematical description of
seismic wave propagation in the Earth and the derivation of
equations that can be solved numerically in an efficient way.
Following a brief review of the elastic wave equation and its
subsidiary conditions in section \ref{S:wave_equation}, we elaborate
on the simulation of anisotropy (section \ref{S:anisotropy}) and
attenuation (section \ref{S:attenuation}). We also cover the
implementation and analysis of absorbing boundaries within this
chapter on the analytical setup, because their properties are
relatively independent from a particular numerical scheme.

\section{The elastic wave equation}\label{S:wave_equation}

The propagation of seismic waves in the Earth can be modelled with
the elastic wave equation
\begin{equation}\label{E:into_waveeq001}
\rho(\w{x})\ddot{\w{u}}(\w{x}, t) - \nabla\cdot\s{\sigma}(\w{x}, t)
= \w{f}(\w{x}, t)\,,
\end{equation}
that relates the displacement field $\w{u}$ to the mass density
$\rho$, the stress tensor $\s{\sigma}$ and an external force density
$\w{f}$ (e.g. Dahlen \& Tromp, 1998; Kennett, 2001; Aki \& Richards, 2002). The elastic wave
equation as presented above is a linearised version of the momentum
balance equation, i.e., of Newton's second law. It is valid under
the assumption that deviations from the reference configuration of
the Earth are small. Furthermore, the rotation of the Earth and its
self-gravitation are omitted. These effects are negligible when
oscillation periods are shorter than $\sim 200$ s. At the surface
$\delta\oplus$ of the Earth $\oplus$ the normal components of the
stress tensor $\s{\sigma}$ vanish:
\begin{equation}\label{E:into_waveeq002}
\s{\sigma}\cdot\w{e}_r |_{\w{x}\in\partial\oplus} = \w{0}\,.
\end{equation}
Equation (\ref{E:into_waveeq002}) is the free surface boundary
condition. Both the displacement field $\w{u}$ and the velocity
field $\w{v}=\dot{\w{u}}$ are required to vanish prior to $t=t_0$
when the external force $\w{f}$ starts to act, i.e.,
\begin{equation}\label{E:into_waveeq003}
\w{u}|_{t<t_0}=\w{v}|_{t<t_0}=\w{0}\,.
\end{equation}
For convenience we will mostly choose $t_0=0$. To obtain a complete
set of equations, the stress tensor $\s{\sigma}$ has to be related
to the displacement field $\w{u}$. It is usually assumed that the
rheology is visco-elastic, meaning that the current stress tensor
$\s{\sigma}$ depends linearly on the history of the strain tensor
$\s{\epsilon}=\frac{1}{2}(\nabla \w{u} + \nabla \w{u}^T)$:
\begin{equation}\label{E:intro_waveeq004}
\s{\sigma}(\w{x}, t) = \int\limits_{-\infty}^{\infty}
\w{\dot{C}}(\w{x}, t-t'):\s{\epsilon}(\w{x}, t')\,dt'\,.
\end{equation}
The fourth order tensor $\w{C}$ is the elastic tensor. In the case
of a perfectly elastic medium the elastic tensor is of the form
$\w{C}(\w{x},t)=\w{C}(\w{x})\,H(t)$, where $H$ is the Heaviside
function. We then have $\s{\sigma}=\w{C}(\w{x}):\s{\epsilon}(\w{x},
t)$. The symmetry of $\s{\epsilon}$, the conservation of angular
momentum and the relation of $\w{C}$ to the internal energy (e.g.
Dahlen \& Tromp, 1998) require that the components of $\w{C}$
satisfy the following symmetry relations:
\begin{equation}\label{E:intro_waveeq005}
C_{ijkl}=C_{klij}=C_{jikl}\,.
\end{equation}
Moreover, the elastic tensor is causal:
\begin{equation}\label{E:into_waveeq006}
\w{C}(t)|_{t<t_0}=\w{0}\,.
\end{equation}
Equation (\ref{E:intro_waveeq004}) is $-$ unlike the wave equation
within its limits of validity $-$ not a fundamental law of physics
but an empirical relation that has been found to describe a wide
range of phenomena very well. It can be regarded as a linearisation
of a more general non-linear constitutive relation. Its validity is,
in this sense, restricted to scenarios where the strain tensor
$\s{\epsilon}$ is a small quantity. The particular choices of
$\w{C}$ and its time dependence will be the subjects of the
following sections on attenuation and anisotropy.

\section{Anisotropy}\label{S:anisotropy} 
Anisotropy is the dependence of the elastic tensor on the orientation of the
coordinate system. Its most direct seismological expression is the
dependence of seismic velocities on the propagation and polarisation
directions of elastic waves. It is thought to play a major role in the Earth's crust and upper mantle.\\
Besides the splitting of shear waves, the Love wave-Rayleigh wave
discrepancy is one of the principal seismic observations that is
directly related to anisotropy: A Love wave and a Rayleigh wave
travelling in the same direction usually exhibit different wave
speeds as a consequence of their different polarisations. This led
to the inclusion of anisotropy with radial symmetry axis in the
global reference model PREM (Dziewonski \& Anderson, 1981). In this
model the anisotropy
is limited to the upper 220 km.\\
Guided by these observations, we decided to implement anisotropy
with radial symmetry axis. For such a medium, there are only 5
independent elastic tensor components that are different from zero.
They can be summarised in a $6\times 6$ matrix (e.g. Babuska \& Cara, 1991):
\begin{align}\label{E:setup15}
&\begin{pmatrix}
C_{r r r r} & C_{r r \phi \phi} & C_{r r \theta \theta} & C_{r r \phi \theta} & C_{r r r \theta} & C_{r r r \phi} \\
C_{\phi \phi r r} & C_{\phi \phi \phi \phi} & C_{\phi \phi \theta \theta} & C_{\phi \phi \phi \theta} & C_{\phi \phi r \theta} & C_{\phi \phi r \phi} \\
C_{\theta \theta r r} & C_{\theta \theta \phi \phi} & C_{\theta \theta \theta \theta} & C_{\theta \theta \phi \theta} & C_{\theta \theta r \theta} & C_{\theta \theta r \phi} \\
C_{\phi \theta r r} & C_{\phi \theta \phi \phi} & C_{\phi \theta \theta \theta} & C_{\phi \theta \phi \theta} & C_{\phi \theta r \theta} & C_{\phi \theta r \phi} \\
C_{r \theta r r} & C_{r \theta \phi \phi} & C_{r \theta \theta \theta} & C_{r \theta \phi \theta} & C_{r \theta r \theta} & C_{r \theta r \phi} \\
C_{r \phi r r} & C_{r \phi \phi \phi} & C_{r \phi \theta \theta} &
C_{r \phi \phi \theta} & C_{r \phi r \theta} & C_{r \phi r \phi}
\end{pmatrix}\notag\\
=&
\begin{pmatrix}
\lambda+2\mu & \lambda+c & \lambda+c & 0 & 0 & 0 \\
\lambda+c & \lambda+2\mu+a & \lambda+a & 0 & 0 & 0 \\
\lambda+c & \lambda+a & \lambda+2\mu+a & 0 & 0 & 0 \\
0 & 0 & 0 & \mu & 0 & 0 \\
0 & 0 & 0 & 0 & \mu+b & 0 \\
0 & 0 & 0 & 0 & 0 & \mu+b
\end{pmatrix}
\end{align}
Love (1892) proposed an alternative parametrisation
\begin{align}\label{E:setup15b}
&\begin{pmatrix}
C_{r r r r} & C_{r r \phi \phi} & C_{r r \theta \theta} & C_{r r \phi \theta} & C_{r r r \theta} & C_{r r r \phi} \\
C_{\phi \phi r r} & C_{\phi \phi \phi \phi} & C_{\phi \phi \theta \theta} & C_{\phi \phi \phi \theta} & C_{\phi \phi r \theta} & C_{\phi \phi r \phi} \\
C_{\theta \theta r r} & C_{\theta \theta \phi \phi} & C_{\theta \theta \theta \theta} & C_{\theta \theta \phi \theta} & C_{\theta \theta r \theta} & C_{\theta \theta r \phi} \\
C_{\phi \theta r r} & C_{\phi \theta \phi \phi} & C_{\phi \theta \theta \theta} & C_{\phi \theta \phi \theta} & C_{\phi \theta r \theta} & C_{\phi \theta r \phi} \\
C_{r \theta r r} & C_{r \theta \phi \phi} & C_{r \theta \theta \theta} & C_{r \theta \phi \theta} & C_{r \theta r \theta} & C_{r \theta r \phi} \\
C_{r \phi r r} & C_{r \phi \phi \phi} & C_{r \phi \theta \theta} &
C_{r \phi \phi \theta} & C_{r \phi r \theta} & C_{r \phi r \phi}
\end{pmatrix}\notag\\
=&
\begin{pmatrix}
C_L & F_L & F_L & 0 & 0 & 0 \\
F_L & A_L & A_L-2N_L & 0 & 0 & 0 \\
F_L & A_L-2N_L & A_L & 0 & 0 & 0 \\
0 & 0 & 0 & N_L & 0 & 0 \\
0 & 0 & 0 & 0 & L_L & 0 \\
0 & 0 & 0 & 0 & 0 & L_L
\end{pmatrix}
\end{align}
We chose to use the parametrisation from equation (\ref{E:setup15})
because it easily allows us to model isotropy by simply setting
$a=b=c=0$. Remember, that all components of the elastic tensor are
functions of time, $t$, and space, $\w{x}$. Density, $\rho$ and the
elastic parameters $\lambda, \mu, a, b$ and $c$ can be related to
the wave speeds of SH, SV, PH and PV waves:
\begin{equation}\label{E:svsh17}
\vsv=\sqrt{\frac{\mu+b}{\rho}}\,,\quad
\vsh=\sqrt{\frac{\mu}{\rho}}\,,\quad
\vpv=\sqrt{\frac{\lambda+2\mu}{\rho}}\,,\quad
\vph=\sqrt{\frac{\lambda+2\mu+a}{\rho}}\,.
\end{equation}
The velocities of SH, SV, PH and PV waves specify only two of the
three additional elastic parameters necessary for anisotropy with
radial symmetry axis, namely $a$ and $b$. Also radially propagating
S waves do not allow us to find $c$, because they propagate with the
velocity $\sqrt{(\mu+b)/\rho}$, just as SV waves. In fact, using
plane waves, $c$ can be determined only from P waves that do not
travel in exactly radial or horizontal directions. Following
Takeuchi \& Saito (1972) and Dziewonski \& Anderson (1981) we
incorporate $c$ into a dimensionless parameter
\begin{equation}\label{E:pvph007}
\eta:=(\lambda+c)/(\lambda+a)\,.
\end{equation}

\section{Attenuation}\label{S:attenuation}

The numerical implementation of attenuation is largely motivated by
technical convenience and not so much by the true physics of seismic
wave attenuation in the interior of the Earth. In our analysis we closely follows Robertsson et al. (1994).\\
Assume that $\sigma, C$ and $\epsilon$ are representative of some
particular components of $\s{\sigma}, \w{C}$ and $\s{\epsilon}$,
respectively. Then a scalar version of the stress-strain relation is
given by
\begin{equation}\label{E:setup16}
\dot{\sigma}(t) = (\dot{C}*\dot{\epsilon})(t) =
\int\limits_{-\infty}^{\infty} \dot{C}(t-t') \dot{\epsilon}(t')\,
dt'\,.
\end{equation}
The spatial dependence has been omitted for brevity. As already
discussed, we choose the stress relaxation function or elastic
tensor component $C$ to be that of a superposition of $N$ standard
linear solids, i.e.,
\begin{equation}\label{E:setup17}
C(t) := C_r \left[ 1- \frac{1}{N} \sum_{p=1}^{N} \left( 1-
\frac{\tau_{\epsilon p}}{\tau_{\sigma p}}\right)\,e^{-t/\tau_{\sigma
p}}\right] \, H(t)\,,
\end{equation}
where $\tau_{\epsilon p}$ and $\tau_{\sigma p}$ are the strain and
stress relaxation times of the $p$th standard linear solid,
respectively. The symbol $H$ denotes the Heaviside function and
$C_r$ is the relaxed modulus. Equation (\ref{E:setup17}) is very
general so that  different sets of relaxation times can give almost
the same relaxation function $C(t)$. To reduce this subjectively
undesirable non-uniqueness we limit the number of free parameters.
Following the $\tau$-method introduced by Blanch et al. (1995) we
determine the relaxation times $\tau_{\epsilon p}$ by defining a
dimensionless variable $\tau$ through
\begin{equation}\label{E:setup18}
\tau := \frac{\tau_{\epsilon p}}{\tau_{\sigma p}} -1\, .
\end{equation}
This gives
\begin{equation}\label{E:setup19}
C(t) = C_r \left[ 1+\frac{\tau}{N} \sum_{p=1}^{N} e^{-t/\tau_{\sigma
p}} \right]\, H(t)\,.
\end{equation}
Differentiating (\ref{E:setup19}) with respect to $t$ and
introducing the result into equation (\ref{E:setup16}) yields
\begin{equation}\label{E:setup20}
\dot{\sigma}(t) = C_r (1+\tau)\,\dot{\epsilon}(t) + C_r
\sum_{p=1}^{N} M_p\,,
\end{equation}
where the memory variables $M_p$ are defined by
\begin{equation}\label{E:setup21}
M_p := -\frac{\tau}{N \tau_{\sigma
p}}\,\int\limits_{-\infty}^{\infty} e^{-(t-t')/\tau_{\sigma p}}\,
H(t-t')\,\dot{\epsilon}(t')\, dt'\,.
\end{equation}
The differentiation of (\ref{E:setup21}) with respect to time yields
a set of first-order differential equations for the memory
variables:
\begin{equation}\label{E:setup22}
\dot{M_p} = -\frac{\tau}{N \tau_{\sigma p}} \, \dot{\epsilon} -
\frac{1}{\tau_{\sigma p}}\, M_p\,.
\end{equation}
Anelasticity can thus be modelled by simultaneously sloving the
momentum equation, a modified stress-strain relation and a set of
$N$ ordinary differential equations for the memory variables $M_p$.
The memory variables are formally independent of the elastic
parameter $C_r$. This formulation, proposed by Moczo \& Kristek
(2005),
gives more accurate results in the case of strong attenuation heterogeneities than the classical formulation by Robertsson et al. (1994).\\
In seismology there has traditionally been more emphasis on the
quality factor Q than on particular stress or strain relaxation
functions. The definition of $Q$ is based on the definition of the
complex modulus
\begin{equation}\label{E:setup28}
\tilde{C}(\nu) := \w{i}\,\nu\,\int_{0}^{\infty} C(t)\,e^{-\w{i}\nu
t}\, dt\,,
\end{equation}
with $\nu:=\omega+\w{i}\gamma$. Then
\begin{equation}\label{E:setup29}
Q(\omega) :=
\frac{\frak{Re}\,\tilde{C}(\omega)}{\frak{Im}\,\tilde{C}(\omega)}\,.
\end{equation}
When $\frak{Im}\,\tilde{C}(\omega) \ll \frak{Re}\,\tilde{C}(\omega)$
then Q can be interpreted in terms of the maximum elastic energy
$E_{max}$ and the energy that is dissipated per cycle, $E_{diss}$:
\begin{equation}\label{E:setup30}
E_{diss} / E_{max} = 4\pi Q^{-1}\,.
\end{equation}
For our stress relaxation function defined in equation
(\ref{E:setup19}) we find
\begin{equation}\label{E:setup31}
Q(\omega) = \frac{\sum_{p=1}^{N} \left( 1+ \frac{\omega^2
\tau_{\sigma p}^2\tau}{1+\omega^2 \tau_{\sigma p}^2} \right)}{
\sum_{p=1}^{N} \left( \frac{\omega \tau_{\sigma p}
\tau}{1+\omega^2\tau_{\sigma p}^2} \right)}\,.
\end{equation}
Generalising equations (\ref{E:setup20}) and (\ref{E:setup22}) to
the case of a three-dimensional and anisotropic medium with radial
symmetry axis is straightforward. The component-wise stress-strain
relation in the absence of dissipation is given by the following set
of equations:
\begin{subequations}\label{E:setup32}
\begin{align}
\sigma_{rr}&=(\kappa-2\mu/3)(\epsilon_{rr}+\epsilon_{\theta\theta}+\epsilon_{\phi\phi})+2\mu\epsilon_{rr}+c(\epsilon_{\theta\theta}+\epsilon_{\phi\phi})\\
\sigma_{\phi\phi}&=(\kappa-2\mu/3)(\epsilon_{rr}+\epsilon_{\theta\theta}+\epsilon_{\phi\phi})+2\mu\epsilon_{\phi\phi}+c\epsilon_{rr}+a(\epsilon_{\theta\theta}+\epsilon_{\phi\phi})\\
\sigma_{\theta\theta}&=(\kappa-2\mu/3)(\epsilon_{rr}+\epsilon_{\theta\theta}+\epsilon_{\phi\phi})+2\mu\epsilon_{\theta\theta}+c\epsilon_{rr}+a(\epsilon_{\theta\theta}+\epsilon_{\phi\phi})\\
\sigma_{r\phi}&=2(\mu+b)\epsilon_{r\phi}\\
\sigma_{r\theta}&=2(\mu+b)\epsilon_{r\theta}\\
\sigma_{\phi\theta}&=2\mu\epsilon_{\phi\theta}
\end{align}
\end{subequations}
We introduced the bulk modulus $\kappa=\lambda+\frac{2}{3}\mu$
because it is $-$ in contrast to the Lam\'{e} parameter $\lambda$ $-$
physically interpretable.\footnote{\textsf{There is, to the best of
my knowledge, no physical interpretation of $\lambda$. The
unphysical nature of this parameter becomes most apparent through
the fact that the $Q$ factor associated with $\lambda$, denoted by
$Q_\lambda$, can be negative for positive $Q_\mu$ and $Q_\kappa$.
Thus, if there were a physical process, e.g. an elastic wave, that
depended only on $\lambda$ and $\rho$, then this process would go
hand in hand with a continuously growing elastic energy.}} The
transition to the dissipative medium is now made by analogy:
\begin{subequations}\label{E:setup33}
\begin{align}
\dot{\sigma}_{rr}&=\left[\kappa_r
(1+\tau_\kappa)-\frac{2}{3}\mu_r(1+\tau_\mu)\right]\,(\dot{\epsilon}_{rr}+\dot{\epsilon}_{\theta\theta}+\dot{\epsilon}_{\phi\phi})+2\mu_r(1+\tau_\mu)\dot{\epsilon}_{rr}+c(\dot{\epsilon}_{\theta\theta}+\dot{\epsilon}_{\phi\phi})\notag\\
&+\kappa_r\sum_{p=1}^N \left(
K_{p}^{rr}+K_{p}^{\theta\theta}+K_{p}^{\phi\phi}\right)
-\frac{2}{3}\mu_r \sum_{p=1}^{N} \left(
M_p^{\theta\theta}+M_p^{\phi\phi} \right) + \frac{4}{3} \mu_r
\sum_{p=1}^{N} M_p^{rr}\\
\dot{\sigma}_{\phi\phi}&=\left[\kappa_r
(1+\tau_\kappa)-\frac{2}{3}\mu_r(1+\tau_\mu)\right]\,(\dot{\epsilon}_{rr}+\dot{\epsilon}_{\theta\theta}+\dot{\epsilon}_{\phi\phi})+2\mu_r(1+\tau_\mu)\dot{\epsilon}_{\phi\phi}+c\dot{\epsilon}_{rr}+a(\dot{\epsilon}_{\theta\theta}+\dot{\epsilon}_{\phi\phi})\notag\\
&+\kappa_r\sum_{p=1}^N \left(
K_{p}^{rr}+K_{p}^{\theta\theta}+K_{p}^{\phi\phi}\right)-\frac{2}{3}\mu_r
\sum_{p=1}^{N}
\left(M_p^{rr}+M_p^{\theta\theta}\right)+\frac{4}{3}\mu_r
\sum_{p=1}^{N} M_p^{\phi\phi}\\
\dot{\sigma}_{\theta\theta}&=\left[\kappa_r
(1+\tau_\kappa)-\frac{2}{3}\mu_r(1+\tau_\mu)\right]\,(\dot{\epsilon}_{rr}+\dot{\epsilon}_{\theta\theta}+\dot{\epsilon}_{\phi\phi})+2\mu_r(1+\tau_\mu)\dot{\epsilon}_{\theta\theta}+c\dot{\epsilon}_{rr}+a(\dot{\epsilon}_{\theta\theta}+\dot{\epsilon}_{\phi\phi})\notag\\
&+\kappa_r\sum_{p=1}^N \left(
K_{p}^{rr}+K_{p}^{\theta\theta}+K_{p}^{\phi\phi}\right)-\frac{2}{3}\mu_r
\sum_{p=1}^{N} \left(M_p^{rr}+M_p^{\phi\phi}\right)+\frac{4}{3}\mu_r
\sum_{p=1}^{N} M_p^{\theta\theta}\\
\dot{\sigma}_{r\phi}&=2\mu_r(1+\tau_\mu)\dot{\epsilon}_{r\phi}+2b\dot{\epsilon}_{r\phi}+2\mu_r\sum_{p=1}^{N}
M_p^{r\phi}\\
\dot{\sigma}_{r\theta}&=2\mu_r(1+\tau_\mu)\dot{\epsilon}_{r\theta}+2b\dot{\epsilon}_{r\phi}+2\mu_r\sum_{p=1}^{N}
M_p^{r\theta}\\
\dot{\sigma}_{\phi\theta}&=2\mu_r(1+\tau_\mu)\dot{\epsilon}_{\phi\theta}+2\mu_r
\sum_{p=1}^{N} M_p^{\phi\theta}
\end{align}
\end{subequations}
For equations (\ref{E:setup33}) we assumed that the parameters $a,
b$ and $c$ are not involved in the dissipation of elastic energy.
The memory variables associated with $\mu$ $-$ denoted by $M_p^{ij}$
$-$ and the memory variables associated with $\kappa$ $-$ denoted by
$K_p^{ij}$ $-$ are governed by the first-order differential
equations
\begin{align}\label{E:setup34}
\dot{M}_p^{ij}&=-\frac{\tau_\mu}{N\tau_{\sigma p, \mu}}
\dot{\epsilon}_{ij} - \frac{1}{\tau_{\sigma p,\mu}} M_{p}^{ij}\\
\dot{K}_p^{ij}&=-\frac{\tau_\kappa}{N\tau_{\sigma p, \kappa}}
\dot{\epsilon}_{ij} - \frac{1}{\tau_{\sigma p,\kappa}} K_{p}^{ij}
\end{align}
The above description of anelasticity is computationally inexpensive
compared to the spatial discretisation of the equations of motion
which account for the largest portion of the computational costs.

\section{Absorbing boundaries}\label{S:absbound_theo}

Restricting the considered spatial domain to only a part of the true physical
domain in the interest of computational efficiency, introduces
unrealistic boundaries. If not treated adequately, the reflections
from the artificial boundaries dominate the numerical error. The
most widely used solutions for this problem fall into two
categories: \emph{absorbing boundary conditions} and \emph{absorbing
boundary layers}.\\
Absorbing boundary conditions are usually based on paraxial
approximations of the wave equation. Early applications of this
technique to finite-difference modelling can be found in Clayton \&
Engquist (1977) or in Stacey (1988). Along the artificial boundary the wave
equation is replaced by one of its paraxial approximations of order
$n$, typically not higher than one or two. The reflection
coefficient then behaves as $(\sin \phi)^n$, where $\phi$ is the
angle of incidence. Absorbing boundary conditions therefore become
inefficient for large angles of incidence in general and for surface
waves in particular. Moreover, they suffer from numerical stability
problems.\\
Absorbing boundary layers are regions near the unphysical boundaries
where the wave field is artificially attenuated in order to prevent
reflections. Cerjan et al. (1985) proposed to multiply the wave
field in each time step with a Gaussian damping function. While this
technique proves to be efficient for finite-difference methods, it
leads to unacceptably large boundary layers when high-order methods
such as the spectral-element method are used. Robertsson et al.
(1994) suggested that the absorption could be improved through
physical dissipation, i.e., a very low $Q$ inside the boundary
layers. This, however, leads to reflections from the boundary layer
itself because low $Q$ values effectively change the elastic
properties of the medium.\\
A comparatively efficient boundary layer technique was introduced by
B\'{e}renger (1994), who proposed to modify the electrodynamic wave
equation inside a \emph{perfectly matched layer} (PML) such that the
solutions decay exponentially with distance, without producing
reflections from the boundary between the regular medium and the
PML. Though originally designed for first-order systems of
differential equations - such as the elastodynamic equations in
stress-velocity formulation (see e.g. Collino \& Tsogka, 2001 or
Festa \& Vilotte, 2005) - it has been shown that the method can be
extended to second-order systems (Komatitsch \& Tromp, 2003). The
classical PML approach leads to a new and generally larger system of
differential equations. Therefore, existing codes have to be
substantially modified. This complication can be avoided by using
\emph{anisotropic perfectly matched layers} (APML), studied for
example by Teixeira \& Chew (1997) and Zheng \& Huang (1997). \SES implements the APML, which are described in the following paragraphs.\\[5pt]
The elastic wave equation in the frequency domain is given by
\begin{subequations}\label{E:apml001}
\begin{align}
\w{i}\omega \rho \w{v}(\w{x},\omega) &= \nabla\cdot\s{\sigma}(\w{x},
\omega) + \w{f}(\w{x},\omega)\,,\label{E:apml001a}\\
\w{i}\omega \s{\sigma}(\w{x},\omega) &=
\s{\Xi}(\w{x},\omega):\nabla \w{v}(\w{x},\omega)\,,\label{E:apml001b}\\
\w{i}\omega \w{u}(\w{x},\omega) &=
\w{v}(\w{x},\omega)\,.\label{E:apml001c}
\end{align}
\end{subequations}
We introduced the symbol $\s{\Xi}$ for the rate of relaxation tensor
$\dot{\w{C}}$ in equations (\ref{E:apml001}). In a homogeneous and
isotropic medium and when external forces are absent, it is known to
have plane wave solutions of the form
\begin{equation}\label{E:apml002}
\w{u}(\w{x},\omega)=\w{A}\,e^{-\w{i}\w{k}(\omega)\cdot \w{x}}\,,
\end{equation}
with either
\begin{subequations}
\begin{align}
&\w{A}\,\, ||\,\, \w{k}(\omega)\quad\text{and}\quad
||\w{k}(\omega)||^2=\frac{\rho\omega^2}{\lambda+2\mu}=\frac{\omega^2}{\vp^2}\,\quad&&\text{(P-wave)}\,\quad\text{or}\,\label{E:apml003}\\
&\w{A}\perp\w{k}(\omega)\quad\text{and}\quad
||\w{k}(\omega)||^2=\frac{\rho\omega^2}{\mu}=\frac{\omega^2}{\vs^2}\,\quad&&\text{(S-wave)}\,.\label{E:apml004}
\end{align}
\end{subequations}
Our goal is to modify (\ref{E:apml001}) such that it allows for
plane wave solutions that decay exponentially with distance, say for
example, in increasing $z$ direction. For this, the cartesian
coordinate $z$ is replaced by a new variable
$\tilde{z}:=\tilde{z}(\w{x},\gamma_z(\w{x}))$, such that the
resulting solutions are
\begin{equation}\label{E:apml005}
\w{u}(\w{x}, \omega)=
\w{A}\,e^{-\w{i}\w{k}(\omega)\cdot\w{x}}\,e^{-f(\gamma_z) z}\,,
\end{equation}
for $z>0$ and some function $f$ that depends on the particular
choice of $\gamma_z$. Choosing $f(\gamma_z)|_{z<0}=0$ will leave the
solution in $z<0$ unchanged. We use the coordinate transformation
\begin{equation}\label{E:apml006}
\tilde{z}:=\int\limits_{0}^{z} \gamma_z(x, y, z') dz'\,.
\end{equation}
Outside the perfectly matched layer, that is in our case for $z<0$,
we require $\tilde{z}=z$ or equivalently $\gamma_z(\w{x})=1$. To
obtain exponentially decaying plane waves also in the remaining
coordinate directions, the transformations $x\to\tilde{x}$ and
$y\to\tilde{y}$ are defined analogously. Replacing the derivatives
$\partial_x$, $\partial_y$ and $\partial_z$ in (\ref{E:apml001a}) by
the derivatives $\partial_{\tilde{x}}$, $\partial_{\tilde{y}}$ and
$\partial_{\tilde{z}}$, gives the following set of equations,
\begin{subequations}\label{E:apml007}
\begin{align}
\w{i}\omega\rho v_x &= \gamma_x^{-1} \partial_x \sigma_{xx} +
\gamma_y^{-1} \partial_y \sigma_{yx} + \gamma_z^{-1} \partial_z
\sigma_{zx}\,,\label{E:apml007a}\\
\w{i}\omega\rho v_y &= \gamma_x^{-1} \partial_x \sigma_{xy} +
\gamma_y^{-1} \partial_y \sigma_{yy} + \gamma_z^{-1} \partial_z
\sigma_{zy}\,,\label{E:apml007b}\\
\w{i}\omega\rho v_z &= \gamma_x^{-1} \partial_x \sigma_{xz} +
\gamma_y^{-1} \partial_y \sigma_{yz} + \gamma_z^{-1} \partial_z
\sigma_{zz}\,.\label{E:apml007c}
\end{align}
\end{subequations}
Note that the divergence in equation (\ref{E:apml001a}) is
interpreted as a left-divergence in equations (\ref{E:apml007}).
Multiplication by $\gamma_x \gamma_y \gamma_z$ yields
\begin{subequations}\label{E:apml008}
\begin{align}
\w{i}\omega \gamma_x \gamma_y \gamma_z v_x &= \gamma_y \gamma_z
\partial_x \sigma_{xx} + \gamma_x \gamma_z \partial_y \sigma_{yx} +
\gamma_x \gamma_y \partial_z \sigma_{zx}\,,\label{E:apml008a}\\
\w{i}\omega \gamma_x \gamma_y \gamma_z v_y &= \gamma_y \gamma_z
\partial_x \sigma_{xy} + \gamma_x \gamma_z \partial_y \sigma_{yy} +
\gamma_x \gamma_y \partial_z \sigma_{zy}\,,\label{E:apml008b}\\
\w{i}\omega \gamma_x \gamma_y \gamma_z v_z &= \gamma_y \gamma_z
\partial_x \sigma_{xz} + \gamma_x \gamma_z \partial_y \sigma_{yz} +
\gamma_x \gamma_y \partial_z \sigma_{zz}\,.\label{E:apml008c}
\end{align}
\end{subequations}
The products $\gamma_i \gamma_j$ on the left-hand sides of
(\ref{E:apml008}) can be placed under the differentiation if
$\gamma_x$ depends only on $x$, $\gamma_y$ only on $y$ and
$\gamma_z$ only on $z$. We may then write the new version of
(\ref{E:apml001a}) as
\begin{equation}\label{E:apml009}
\w{i}\omega \gamma_x \gamma_y \gamma_z \w{v} = \nabla\cdot
\s{\sigma}^{\text{pml}}\,,
\end{equation}
with the definition
\begin{equation}\label{E:apml010}
\s{\sigma}^{\text{pml}} := \gamma_x \gamma_y \gamma_z \left(
  \begin{array}{ccc}
    \gamma_x^{-1} & 0 & 0 \\
    0 & \gamma_y^{-1} & 0 \\
    0 & 0 & \gamma_z^{-1} \\
  \end{array}
\right) \cdot \s{\sigma} = \gamma_x \gamma_y \gamma_z
\s{\Lambda}\cdot\s{\sigma}\,.
\end{equation}
Since (\ref{E:apml001a}) and (\ref{E:apml009}) are very similar, the
anisotropic perfectly matched layers can be implemented with
relative ease by slightly modifying pre-existing codes. It is
noteworthy at this point that $\s{\sigma}^{\text{pml}}$ is in
general not symmetric.\\
In the next step we consider the constitutive relation. Based on
(\ref{E:apml001b}) and (\ref{E:apml001c}) we find
\begin{equation}\label{E:apml011}
\sigma_{ij}^{\text{pml}} = \gamma_x \gamma_y \gamma_z \gamma_i^{-1}
\sigma_{ij} = \gamma_x \gamma_y \gamma_z \gamma_i^{-1}
\sum_{k,l=1}^{3} \Xi_{ijkl} \partial_{x_k} u_l\,.
\end{equation}
Again replacing $\partial_x$, $\partial_y$ and $\partial_z$ by
$\partial_{\tilde{x}}$, $\partial_{\tilde{y}}$ and
$\partial_{\tilde{z}}$, yields
\begin{equation}\label{E:apml012}
\sigma_{ij}^{\text{pml}}=  \gamma_x \gamma_y \gamma_z \gamma_i^{-1}
\sum_{k,l=1}^{3} \Xi_{ijkl} \gamma_k^{-1} \partial_{x_k} u_l\,.
\end{equation}
For convenience we define a new strain tensor
$\s{\epsilon}^{\text{pml}}$ as
\begin{equation}\label{E:apml013}
\epsilon_{kl}^{\text{pml}} :=  \gamma_x \gamma_y \gamma_z
\gamma_k^{-1} \partial_{x_k} u_l\,,
\end{equation}
which finally leads to our modified version of the wave equation:
\begin{subequations}\label{E:apml014}
\begin{align}
\w{i}\omega \rho \gamma_x \gamma_y \gamma_z v_i &= \sum_{j=1}^{3}
\partial_{x_j} \sigma_{ji}^{\text{pml}}\,,\label{E:apml014a}\\
\gamma_i \sigma_{ij}^{\text{pml}} &= \sum_{k,l=1}^{3} \Xi_{ijkl}
\epsilon_{kl}^{\text{pml}}\,,\label{E:apml014b}\\
\epsilon_{kl}^{\text{pml}} &=  \gamma_x \gamma_y \gamma_z
\gamma_k^{-1} \partial_{x_k} u_l\,.\label{E:apml014c}
\end{align}
\end{subequations}
The source term in (\ref{E:apml014}) is omitted because the PML
region will usually be chosen such that the sources are inside the
regular medium and not inside the PML.\\[10pt]
To demonstrate that exponentially decaying plane waves are indeed
solutions of (\ref{E:apml014}) we consider a homogeneous and
isotropic medium. For simplicity, we restrict the analysis to the
case $\gamma_x=\gamma_y=1$, $\gamma_z = const \neq 1$. Assuming a
plane wave solution of the form
\begin{equation}\label{E:apml015}
\w{u}(\w{x},\omega)= \w{A}\,e^{-\w{i}(k_x x + k_z z ) }\,,
\end{equation}
the problem reduces to finding the dispersion relation
$\w{k}=\w{k}(\omega)$. The Christoffel equation resulting from the
combination of (\ref{E:apml014}) with (\ref{E:apml015}) is
\begin{equation}\label{E:apml016}
\left(
  \begin{array}{ccc}
    \vp^2 k_x^2 +\vs^2 k_z^2 \gamma_z^{-2} & 0 & (\vp^2-\vs^2)(k_x k_z \gamma_z^{-1}) \\
    0 & \vs^2 (k_x^2 + k_z^2 \gamma_z^{-2}) & 0 \\
    (\vp^2-\vs^2)(k_x k_z \gamma_z^{-1}) & 0 & \vp^2 k_z^2 \gamma_z^{-2} + \vs^2 k_x^2 \\
  \end{array}
\right) \left(
  \begin{array}{c}
    A_x \\
    A_y \\
    A_z \\
  \end{array}
\right)
 = \omega^2 \left(
                     \begin{array}{c}
                       A_x \\
                       A_y \\
                       A_z \\
                     \end{array}
                   \right)\,.
\end{equation}
Non-trivial solutions exist only if either
\begin{subequations}\label{E:apml017}
\begin{equation}\label{E:apml018}
k_x^2+k_z^2 \gamma_z^{-2} = \omega^2 \vp^{-2}\,
\end{equation}
or
\begin{equation}\label{E:apml019}
k_x^2+k_z^2 \gamma_z^{-2} = \omega^2 \vs^{-2}\,
\end{equation}
\end{subequations}
are satisfied. Equations (\ref{E:apml018}) and (\ref{E:apml019}) are
the dispersion relations inside the PML. The resulting plane wave
solutions are
\begin{subequations}\label{E:apml020}
\begin{equation}
\w{u}_s(\w{x},\omega)=\w{A}_s\,e^{-\w{i} \omega
(x\,\sin\phi+\gamma_z z\,\cos\phi)/\vs}\,,\quad \w{A}_s\perp (k_x,
0, k_z\gamma_z^{-1})^T\,,
\end{equation}
for the S wave and
\begin{equation}
\w{u}_p(\w{x},\omega)=\w{A}_p\,e^{-\w{i} \omega
(x\,\sin\phi+\gamma_z z\,\cos\phi)/\vp}\,,\quad \w{A}_p\,||\, (k_x,
0, k_z\gamma_z^{-1})^T\,,
\end{equation}
\end{subequations}
for the P wave. The variable $\phi$ denotes the angle of incidence.
One may now define the coordinate stretching variable $\gamma_z$.
There are in principle no restrictions other than $\gamma_z=1$
outside the PML. We use
\begin{equation}\label{E:apml021}
\gamma_z(\w{x}):=1+\frac{a_z}{\w{i}\omega}\,.
\end{equation}
Inserting (\ref{E:apml021}) into (\ref{E:apml020}) yields
\begin{equation}\label{E:apml022}
\w{u}_{s/p}(\w{x},\omega)=\w{A}_{s/p}\,e^{-\w{i} \omega (x\,\sin\phi
+
z\,\cos\phi)/v_\text{\tiny{S/P}}}\,e^{-(a_z\,z\cos\phi)/v_\text{\tiny{S/P}}}\,.
\end{equation}
The exponential decay is therefore frequency-independent in the case
of incident body waves. A frequency-dependent decay can be expected
for dispersive waves such as surface waves. However, waves with an
angle of incidence, $\phi$, close to $\pm \pi/2$ will decay slower
than waves with an angle of incidence close to $0$. It should be
noted that the example of a homogeneous and isotropic medium with
constant $\gamma_z$ is oversimplified. Since analytical solutions
for more complex models are usually unavailable, the efficiency of
the APML
technique must be tested numerically.\\[5pt]
With the exception of the corners of the
model\footnote{\textsf{There is, to the best of my knowledge, no PML
variant where the corners are treated adequately. In the corners two
or more PMLs overlap, and it is not immediately clear what the
consequences are. In fact, numerical experiments show that the
instability tends to start in the corners. This suggests that PML
methods might be improved substantially through the construction of
absorbing corners where elastic waves do indeed decay.}}, the
damping regions at the $x$, $y$ and $z$ boundaries do not overlap.
We therefore find
\begin{equation}\label{E:apml023}
a_i a_j = a_i^2 \delta_{ij}
\end{equation}
and
\begin{equation}\label{E:apml024}
\gamma_x \gamma_y \gamma_z = \frac{\w{i}\omega + a_x + a_y +
a_z}{\w{i}\omega}\,.
\end{equation}
The resulting equations of motion in the frequency domain are now
\begin{subequations}\label{E:apml025}
\begin{align}
\w{i}\omega\rho\,(\w{i}\omega + a_x + a_y + a_z)\, u_i &=
\sum_{j=1}^{3} \partial_{x_j}
\sigma_{ji}^{\text{pml}}\,,\label{E:apml025a}\\
(\w{i}\omega + a_i)\,\sigma_{ij}^{\text{pml}} &= \sum_{k,l=1}^{3}
\w{i}\omega\,\Xi_{ijkl} \epsilon_{kl}^{\text{pml}}\,,\label{E:apml025b}\\
\w{i}\omega\,\epsilon_{kl}^{\text{pml}} &=(\w{i}\omega + a_x + a_y +
a_z - a_k)\,\partial_{x_k} u_l\,.\label{E:apml025c}
\end{align}
\end{subequations}
Transforming into the time domain, finally yields
\begin{subequations}\label{E:apml026}
\begin{align}
\rho\,\partial_t^2 u_i + \rho\,(a_x+a_y+a_z) \partial_t u_i &=
\sum_{j=1}^{3} \partial_{x_j}
\sigma_{ji}^{\text{pml}}\,,\label{E:apml026a}\\
\partial_t \sigma_{ij}^{\text{pml}} + a_i \sigma_{ij}^{\text{pml}}
&= \sum_{k,l=1}^{3} \Xi_{ijkl}
*\partial_t{\epsilon}_{kl}^{\text{pml}}\,,\label{E:apml026b}\\
\partial_t{\epsilon}_{kl}^{\text{pml}} &= \partial_t \partial_{x_k} u_l +
(a_x+a_y+a_z-a_k)\,\partial_{x_k} u_l\,,\label{E:apml026c}
\end{align}
\end{subequations}
or in tensor notation:
\begin{subequations}\label{E:apml027}
\begin{align}
\rho\,\partial_t^2\w{u}+\rho\,\text{tr}(\w{a})\,\partial_t \w{u} &=
\nabla\cdot\s{\sigma}^{\text{pml}}\,,\label{E:apml027a}\\
\partial_t\s{\sigma}^{\text{pml}} +
\w{a}\cdot\s{\sigma}^{\text{pml}} &=\int\limits_{-\infty}^{t}
\s{\Xi}(t-\tau):\partial_t\s{\epsilon}^{\text{pml}}(\tau)\,d\tau\,,\label{E:apml027b}\\
\partial_t\s{\epsilon}^{\text{pml}} &= \partial_t \nabla\w{u} +
(\w{I}\,\text{tr}(\w{a})-\w{a})\cdot\nabla
\w{u}\,,\label{E:apml027c}
\end{align}
\end{subequations}
where $\w{a}$ is defined as
\begin{equation}\label{E:apml028}
\w{a}:=\left(
         \begin{array}{ccc}
           a_x & 0 & 0 \\
           0 & a_y & 0 \\
           0 & 0 & a_z \\
         \end{array}
       \right)\,.
\end{equation}
The gradients in equation (\ref{E:apml027c}) are left-gradients. The
choice of $\gamma_i$ as inversely proportional to $\omega$ leads to
relatively simple differential equations that can be marched forward
in time explicitly. Different definitions of $\gamma_i$ generally
result in temporal convolutions and therefore lead to
integro-differential equations that are more difficult to solve, at
least in the time domain.
